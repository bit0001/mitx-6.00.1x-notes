\documentclass[a4paper]{article}
\usepackage{amsmath}
\usepackage{amsthm}
\newtheorem{thm}{Theorem}
\usepackage[english]{babel}
\usepackage[utf8]{inputenc}
\usepackage{graphicx}
\usepackage[colorinlistoftodos]{todonotes}
\setlength{\parskip}{8pt}
\setlength{\parindent}{0pt}

\title{Introduction to Python}

\author{}
\date{}
\begin{document}
\maketitle

\section{Introduction}

The purposes of this course are:

\begin{itemize}
  \item Learn computational models of thinking.
  \item Master the art of computational problem solving.
  \item Make computers do what you want them to do.
\end{itemize}

\subsection{What does a computer do?}

Fundamentally, a computer does only two things:

\begin{enumerate}
  \item It performs calculations.
  \item It remember results.
\end{enumerate}

One might be wondering how fast a computer performs calculations, so nowadays
a computer can perform approximately a billion calculations per second.
On the other hand, nowadays computers have a hard drive of nearly 1 TB which
means that a computer can store approximately 1.5 million books of standard
size.

\subsection{What kind of operations does a computer perform?}

Every computer comes with a set of built-in operations. These are typically
primitive arithmetic operations such as addition, multiplicaton, division, and
simple logic operations such as comparing true and false values in order make
decisions.

Because this built-in tools are not enough, in this course one will learn how
to define new calculations, new operations, and give them to the computer, so
it can abstract them, encapsulate them, and treat them as they are primitives.

\subsection{Simple calculations and storage enough?}

It is very important to design good algorithms since simple taks could take
a lot of time, and results could required a lot of space to be storaged.
For example, searching on the world wide web using simple operations could take
5.2 days, or deciding a chess move without an optimal algorithm could take 30
minutes.

Considering the same chess problem, experts suggest that there are
approximately $10^{123}$ different possible games. However, there are only
$10^{80}$ atoms in the observable universe. Hence, it is simply not
possible to store all different possible games in chess.

\subsection{Are there limits?}

Despite its speed and size, a computer has limitations. For instance, some
problems are still too complex such as accurate weather prediction and
cracking encryption schemes, or some ploblems are fundamentally impossible to
compute as the famous Halting problem which basically asks to predict whether
a piece of code will always halt with an answer for any input.

\end{document}
